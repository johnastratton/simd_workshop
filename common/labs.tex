\documentclass[12pt]{article}
\usepackage{amsmath}
\usepackage{amssymb}
\usepackage{graphicx}
\usepackage{multicol}
\usepackage{float}
\usepackage{fancyhdr}
\usepackage{lastpage}
\pagestyle{fancy}

\lhead{Vector Programming Study}
\rhead{\thepage}
\cfoot{Vector Programming}

\renewcommand{\headrulewidth}{0.4pt}
\renewcommand{\footrulewidth}{0.4pt}

\title{Vector Programming Study}
\author{Vikings!}

\begin{document}

\maketitle

\par The goal of these assignments is to help solidify your understanding of both methods of vector programming in a high level language, and hopefully give you an insight into a usability comparison of the two methods of SSE vector programming for Intel chips.
This assignment is being administered as part of a study of comparing the usability and programmer efficiency of the use of these two systems.
Data will be collected regarding your compile/code intervals, the correctness of the assignment at each compile, runtime speed at each compile, and analysis information of the assembly code produced at each compile.

\noindent Some things to keep in mind while you are completing this assignment:

\begin{itemize}
\item You must have \textbf{gcc} and GNU \textbf{make} installed in order to complete this assignment.
\item You must have an SSE capable system to complete this assignment.
To make sure your system is SSE capable, run \textbf{cat /proc/cpuinfo | grep sse} and ensure that there is output.
\item Run the command \textbf{make} in your build directory to compile your project and run the test cases.
There are built in test cases that check your program for correctness and efficiency.
Note any warnings or errors in your runtime, they will give you clues as to what to fix.
\item For vector programming, be especially aware of alignment of vector loads.
\end{itemize}

\end{document}
